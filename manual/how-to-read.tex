\chapter{How to read this manual}

To improve readability, the following concepts or entities discussed in this manual are
highlighted.\\

\begin{description}[labelwidth=.15\textwidth]
    \item[shell specific] Shell commands are \cli{monospaced-lowercase-and-grayed}.
    Environment variables are \envvar{ALL\_CAPS\_DOLLAR\_PREFIXED}.
\begin{commandline}
echo "Example commands are boxed with a gray background."
# Comments are prefixed with a hash mark.
\end{commandline}

    \item[OS specific] File names are \filename{light-blue}, and \directory{/directories/end/with/a/slash/}.
    Package names try to follow the original typesetting and are \package{green and bold}.
\begin{file}
File contents are boxed with a light-blue background.
\end{file}

    \item[\ctg specific] \observable{Obserservables}, \option{options} and \parameter{parameters} are orange
    and adhere to a syntax defined later in this manual. Classes are \class{CamelCaseNoUnderscores}. Methods
    are (almost always) \method{lower\_case\_with\_underscores}.
\begin{sourcecode}
print('Source code is boxed with a light orange back ground.')
# Source code listings have line numbers.
\end{sourcecode}
\end{description}
