%\chapter{Effective Field Theory}
%
%ctg uses various Effective Field Theories (EFTs) below the electroweak scale to
%parametrize the effects of the Standard Model (SM) or New Physics (NP) in flavour-changing
%decays. The relevant EFTs for $b\to s\ell\ell$ decays, and for $b\to \lbrace u, c\rbrace \ell\nu$
%decays are documented in the following.
%
%\section{Effective Field Theory for $b\to s\ell\ell$ transitions}
%
%The effective Lagrangian for $b\to s\ell\ell$ transitions reads:
%\begin{equation}
%    \mathcal{L}^\text{eff} = \frac{4 G_F}{\sqrt{2}} V_{tb}^{} V_{ts}^* \left\lbrace
%        \left[C_1 O_1^c + C_2 O_2^c\right]
%        + \frac{\alpha_e}{4\pi}\left[C_7 O_7 + C_9 O_9 + C_{10} O_{10}\right]
%    \right\rbrace + \order{\frac{V_{ub}^{} V_{us}^*}{V_{tb}^{} V_{ts}^*}, C_{3,\dots,6}, \alpha_s C_8}\,,
%\end{equation}

%--------+---------+---------+---------+---------+---------+---------+---------+
%
%  intro to physics section
%
%--------+---------+---------+---------+---------+---------+---------+---------+

The electro-weak decays of hadrons (mesons and baryons) with masses much smaller
than the electro-weak scale of order of the $W$-boson mass, $m_W$, can be efficiently
described using effective field theories (EFT) in the spirit of the well-known
Fermi theory of the $\beta$-decay. This comprises practically all hadrons containing
light quarks $q = (u,\,d,\,s,\,c,\,b)$. In this context, specific flavour-changing
higher-dimensional ($dim > 4$) operators arise in the standard model (SM) and
it's extensions, accompanied by effective couplings (Wilson coefficients) giving
rise to the generic structure of the effective Lagrangian
\begin{align}
  {\cal L}_{\text{EFT}} &
  = {\cal L}_{\text{QCD} \times \text{QED}} (\mbox{light particles})
  + \sum_i \wilson{i}(\mu) \op{i} + \mbox{h.c.} + \ldots \,.
\end{align} 
The first term describes the $SU(3)_c \times U(1)_\text{em}$
gauge interactions of all light quarks, $q$, and leptons, $\ell = (e,\,\mu,\,\tau)$,
with QCD and QED gauge bosons. The second part are the aforementioned operators
$\op{i}$ and effective couplings $\wilson{i}$, where the latter depend on a
factorization scale $\mu_\text{low}$ that is usually of the order of the mass of the
decaying hadron. The operators are composed out of light degrees of freedom, i.e. fermions
$q$ and $\ell$, as well as $SU(3)_c$ and $U(1)_\text{em}$ gauge bosons. Beyond
the SM, it is conceivable that in principle additional light degrees of freedom
exist, which however must have escaped direct detection so far.
The Wilson coefficients are suppressed by inverse powers of the electroweak
scale or some new physics scale, depending on the dimension of their operators.
Finally, the dots denote higher-dimensional operators that have been neglected.
In practical applications, assuming the SM, they are suppressed at least by
$m_b^2/m_W^2 \sim 0.004$, where $m_b$ denotes the bottom-quark mass.

The EFT framework is sufficient to describe interactions at and below the scale
$\mu_\text{low}$, including hadronic binding effects due to strong interactions
as well as electro-magnetic virtual and real corrections to observables. The
implementation of according observables in ctg is thus based on universal EFT 
Lagrangians. All respective short-distance interactions above $\mu_\text{low}$ are
fixed by the structure of the operators and the values of the Wilson coefficients.
In the spirit of Fermi, the according Wilson coefficients can be viewed as unknowns
to be determined from experiment. This so-called {\em model-independent} approach
implies the independence of all Wilson coefficients and correlation among observables
arise from the assumption of which operators are included, i.e. have non-vanishing
Wilson coefficients at the scale $\mu_\text{low}$ (or some other). 

The latter point
is important in view of operator mixing (under QCD and QED), which is governed by
the anomalous dimension matrices (ADM) of the involved operators. In the case of
operator mixing, vanishing Wilson coefficients at some scale $\mu_0$ can become 
non-zero at some other scale $\mu_1$, depending on their mixing and the initial
conditions of Wilson coefficients involved in the mixing. This is summarized
by the general solution of the renormalisation group equation (RGE)
\begin{align}
  \wilson{i}(\mu_1) & = \sum_j [U(\mu_1, \mu_0)]_{ij} \wilson{j}(\mu_0)\,.
\end{align}
The evolution matrix $U$ depends on the ADM's of the operators, the strong
and electro-magnetic couplings, $\alpha_s$ and $\alpha_e$ respectively, and
their respective RG-functions (beta-functions). Throughout Wilson coefficients
and couplings are $\overline{\text{MS}}$-renormalized quantities.

In the following we collect the conventions of the effective theories implemented
in ctg for hadron decays based on quark transitions
\begin{align}
  b & \to (d,\, s), & \ldots
\end{align}
For an introduction to the topic, the reader is referred to the exhaustive review
articles \cite{Buchalla:1995vs, Buras:1998raa}. 

The following abbreviations will be often used below for products of elements
of the CKM quark mixing matrix appearing in $b$-quark decays
\begin{align}
  \lambda_U^{(D)} & = V_{Ub}^{} V_{UD}^* \,, &
  D & = (d,\,s), \, &
  U & = (u,\,c,\,t) \,.
\end{align}

%--------+---------+---------+---------+---------+---------+---------+---------+
%
%  b -> s  EFT
%
%--------+---------+---------+---------+---------+---------+---------+---------+

\section{$|\Delta B| = |\Delta S| = 1$}

In this section we summarize the convention of the EFT for $|\Delta B| = |\Delta S| = 1$
decays covering transitions
\begin{align*}
  b & \to s + (\bar{u}u,\, \bar{c}c)           && \mbox{current-current} \,,
\\
  b & \to s + \bar{q}q                         && \mbox{QCD \& QED penguin} \,,
\\
  b & \to s + (\gamma,\, \mbox{gluon})         && \mbox{electro- \& chromo-magnetic dipole} \,,
\\
  b & \to s + (\bar{\ell}\ell,\, \bar{\nu}\nu) && \mbox{semi-leptonic} \,,
\end{align*}
where the classification corresponds to the origin of the operators from 
decoupling $W$ and $Z$ bosons and the top quark in the SM. The basis contains several
blocks that are however related via operator mixing and renders the RGE of the
Wilson coefficients non-trivial. The results for $|\Delta B| = |\Delta D| = 1$
transitions are obtained by the replacement $s\to d$.
 
We start with the operators generated in the SM, where the initial Wilson coefficients
and ADM's are known at several orders in perturbation theory.
The most appropriate choice of basis for higher order QCD calculations of ADM's was
given in \cite{Chetyrkin:1996vx, Chetyrkin:1997gb} with according extension for
QED corrections in \cite{Bobeth:2003at, Huber:2005ig}. The Lagrangian
\begin{equation}
\begin{aligned}
  {\cal L}_\text{EFT} &
  = {\cal L}_{\text{QCD} \times \text{QED}} (q;\, \ell)
  + \frac{4 G_F}{\sqrt{2}} \lambda_{u}^{(s)} \sum_{i=1}^{2} 
     \wilson{i} (\op{i}^c - \op{i}^u)
\\
  & + \frac{4 G_F}{\sqrt{2}} \lambda_{t}^{(s)} \left[
      \sum_{i=1}^{2}  \wilson{i} \op{i}^c
    + \sum_{i=3}^{10} \wilson{i} \op{i}
    + \sum_{i=3}^{6}  \wilson{iQ} \op{iQ}
    + \wilson{b}\op{b} \right] + \mbox{h.c.} \,.
\end{aligned}
\end{equation}
incorporates unitarity of the CKM matrix  $\lambda_u^{(s)} + \lambda_c^{(s)} + 
\lambda_t^{(s)} = 0$. All Wilson coefficients $\wilson{i}$ are evaluated at
$\mu_\text{low} \sim m_b$. The up-quark
sector $\sim \lambda_u^{(s)}$ is doubly-Cabibbo suppressed for $b\to s$
transitions and leads to tiny CP-asymmetries in the SM.

The current-current ($U = u,\, c$) operators are defined as 
\begin{equation}
\begin{aligned}
  \op{1}^U & = (\bar{s} \gamma_\mu P_L T^a U) (\bar{U} \gamma^\mu P_L T^a b) \,, &
  \op{2}^U & = (\bar{s} \gamma_\mu P_L U) (\bar{U} \gamma^\mu P_L b) \,, 
\end{aligned}
\end{equation}
which arise already at tree-level from decoupling the $W$-boson in $b\to s + 
(\bar{u}u,\, \bar{c}c)$ and mix into all other operators, except the
semi-leptonic $\op{10}$. Here and below $P_{L,R} = (1 \mp \gamma_5)/2$ denote
chirality projectors. 

The QCD-penguin operators are
\begin{equation}
\begin{aligned}
  \op{3} & = (\bar{s} \gamma_\mu P_L b)     \sum_q (\bar{q} \gamma^\mu q) \,, &
  \op{5} & = (\bar{s} \gamma_{\mu\nu\rho} P_L b)
             \sum_q (\bar{q} \gamma^{\mu\nu\rho} q) \,,
\\[1mm] 
  \op{4} & = (\bar{s} \gamma_\mu P_L T^a b) \sum_q (\bar{q} \gamma^\mu T^a q) \,, &
  \op{6} & = (\bar{s} \gamma_{\mu\nu\rho} P_L T^a b) 
             \sum_q (\bar{q} \gamma^{\mu\nu\rho} T^a q) \,,
\end{aligned}
\end{equation}
where the sum extends over all $q = (u,\,d,\,s,\,c,\,b)$ and $T^a$ are the
generators of $SU(3)_c$. Products of several gamma matrices have been
abbreviated $\gamma^{\mu\nu\rho} \equiv \gamma^\mu\gamma^\nu \gamma^\rho$. 

Analogous QED-penguin operators are defined as 
\begin{equation}
\begin{aligned}
  \op{3Q} & = (\bar{s} \gamma_\mu P_L b)     \sum_q Q_q (\bar{q} \gamma^\mu q) \,, & 
  \op{5Q} & = (\bar{s} \gamma_{\mu\nu\rho} P_L b)
             \sum_q Q_q (\bar{q} \gamma^{\mu\nu\rho} q) \,, &
\\[1mm]   
  \op{4Q} & = (\bar{s} \gamma_\mu P_L T^a b) \sum_q Q_q (\bar{q} \gamma^\mu T^a q) \,, &
  \op{6Q} & = (\bar{s} \gamma_{\mu\nu\rho} P_L T^a b) 
             \sum_q Q_q (\bar{q} \gamma^{\mu\nu\rho} T^a q) \,,
\end{aligned}
\end{equation}
where $Q_q$ denotes the quark charges as multiples of $e$. Further, an additional
operator has to be considered
\begin{align}
  \op{b} & = -\frac{1}{3}  (\bar{s} \gamma_\mu P_L b)(\bar{b} \gamma^\mu b)
             +\frac{1}{12} (\bar{s} \gamma_{\mu\nu\rho} P_L b)
                           (\bar{b} \gamma^{\mu\nu\rho} b) \,,
\end{align}
receiving contributions from electro-weak boxes. In four dimension it would
correspond to $(\bar{s} \gamma_\mu P_L b)(\bar{b} \gamma^\mu P_L b)$, however
the above form allows to avoid traces with $\gamma_5$ to all orders in QCD.

The electro- and chromo-magnetic dipole operators
\begin{align}
  \tildeop{7} & = \frac{e}{g_s^2} [\bar{s} \sigma^{\mu \nu} 
                  (\overline{m}_b P_R + \overline{m}_s P_L) b] F_{\mu \nu} \,, &
  \tildeop{8} & = \frac{1}{g_s}   [\bar{s} \sigma^{\mu \nu}
                  (\overline{m}_b P_R + \overline{m}_s P_L) T^a b] G_{\mu \nu}^a \,, &  
\end{align}
receive contributions from on-shell photon and gluon penguins. The appearing
quark masses are renormalized in the $\overline{\mbox{MS}}$ scheme. 

In the SM there are two semi-leptonic operators
\begin{align}
  \tildeop{9}^{\ell}  & = \frac{e^2}{g_s^2} (\bar{s} \gamma_\mu P_L b) 
                          (\bar{\ell} \gamma^\mu \ell) \,, &
  \tildeop{10}^{\ell} & = \frac{e^2}{g_s^2} (\bar{s} \gamma_\mu P_L b)
                          (\bar{\ell} \gamma^\mu \gamma_5 \ell) \,,
\end{align}
describing $b\to s + \bar\ell\ell$ transitions. In this case the lepton
charge $Q_\ell$ has been pulled into the definition of the Wilson
coefficient.

{\em note definition of evanescent operators, without whom ADM's and initial Wilson 
coefficients are meaningless ...} 

For the tilde operators the normalization to $(4\pi/g_s)^2$, the QCD coupling, has been chosen
for practical reasons such that the leading SM 1-loop correction to the initial Wilson coefficients
counts formally as a strong correction rather then an electro-magnetic one. The 
initial Wilson coefficients are known up to NNLO in QCD and NLO in EW corrections
\begin{align}
  \wilson{i}(\mu_0) = \dots
\end{align}
($a_i \equiv \alpha_i/(4\pi)$)
For practical applications, however, we use operators without a tilde, which are related via
\begin{align}
    \op{i} = \frac{g_s^2}{(4\pi)^2}\, \tildeop{i}\,.
\end{align}

\begin{table}[t]
\begin{center}
\begin{tabular}{@{} p{.25\linewidth} p{.25\linewidth} @{}}
\toprule
    \textbf{Quantity} & \textbf{Qualified Name}\\
\midrule
    \multicolumn{2}{c}{$i=7,7'$}\\
\midrule
    $\Re C_i(\mu_b)$                 & \verb|b->s::Re{c|$i$\verb|}|\\
    $\Im C_i(\mu_b)$                 & \verb|b->s::Im{c|$i$\verb|}|\\
\midrule
    \multicolumn{2}{c}{$i=9,9',10,10',S,S',P,P',T,T5$}\\
\midrule
    $\Re \wilson[(\ell)]{i}(\mu_b)$  & \verb|b->s|$\ell\ell$\verb|::Re{c|$i$\verb|}|\\
    $\Im \wilson[(\ell)]{i}(\mu_b)$  & \verb|b->s|$\ell\ell$\verb|::Im{c|$i$\verb|}|\\
\midrule
    \multicolumn{2}{c}{$i=1\dots 6, 8,8'$}\\
\midrule
    $C_i(\mu_b)$                   & \verb|b->s::c|$i$\\
\bottomrule
\end{tabular}
\end{center}
\caption{Naming scheme for the $|\Delta B| = |\Delta S| = 1$ operators.}
\label{tab:eft:bsll:naming}
\end{table}

The full basis of semi-leptonic operators is spanned by $\op{9,10}$ as above, and additionally
\begin{align}
  \op{9'}^{\ell}  & = \frac{e^2}{(4\pi)^2} (\bar{s} \gamma_\mu P_R b) 
                      (\bar{\ell} \gamma^\mu \ell) \,, &
  \op{10'}^{\ell} & = \frac{e^2}{(4\pi)^2} (\bar{s} \gamma_\mu P_R b)
                      (\bar{\ell} \gamma^\mu \gamma_5 \ell) \,,\\
  \op{S}^{\ell}   & = \frac{e^2}{(4\pi)^2} (\bar{s} P_R b)
                      (\bar{\ell} \ell) \,, &
  \op{P}^{\ell}   & = \frac{e^2}{(4\pi)^2} (\bar{s} P_R b)
                      (\bar{\ell} \gamma_5 \ell) \,,\\
  \op{S'}^{\ell}  & = \frac{e^2}{(4\pi)^2} (\bar{s} P_L b)
                      (\bar{\ell} \ell) \,, &
  \op{P'}^{\ell}  & = \frac{e^2}{(4\pi)^2} (\bar{s} P_L b)
                      (\bar{\ell} \gamma_5 \ell) \,,\\
  \op{T}^{\ell}   & = \frac{e^2}{(4\pi)^2} (\bar{s} \sigma_{\mu\nu} b)
                      (\bar{\ell} \sigma^{\mu\nu} \ell) \,, &
  \op{T5}^{\ell}  & = \frac{e^2}{(4\pi)^2} (\bar{s} \sigma_{\mu\nu} b)
                      (\bar{\ell} \sigma^{\mu\nu} \gamma_5 \ell) \,.
\end{align}
For studies of NP effects in these operators the values of the respective Wilson coefficients
can be changed within the ctg code at run time. For the changed to take effect, the observables
must be constructed with the option \option{model} set to \texttt{WilsonScan}. In that case, the
coefficients $\wilson{i}$ with $i\in \lbrace 7,7',9,9',10,10',S,S',P,P',T,T5\rbrace$ are treated
as complex-valued parameters.
Since ctg parameters are real-valued only, this implies that real and
imaginary part of these coefficients can be adjusted separately.
The coefficients $\wilson{j}{U}$ with $j=\lbrace 1,2\rbrace$ are presently
$U$-flavour-universal, and are simply considered as two independent coefficients $\wilson{j}$.
The coefficients with $j=\lbrace 1\dots 6, 8\rbrace$ are treated as real-valued parameters.
The naming scheme for these coefficients is listed in \reftab{eft:bsll:naming}.


%--------+---------+---------+---------+---------+---------+---------+---------+
%
%  b -> s  EFT
%
%--------+---------+---------+---------+---------+---------+---------+---------+

\section{$|\Delta B| = |\Delta U| = 1$}

The Lagrangian reads:
\begin{equation}
\begin{aligned}
  {\cal L}_\text{EFT} &
  = {\cal L}_{\text{QCD} \times \text{QED}} (q;\, \ell) - \frac{4 G_F}{\sqrt{2}} V_{ub}^\text{eff} \left[
      \sum_{X}  \wilson{X} \op{X}
    \right] \,.
\end{aligned}
\end{equation}
Note the change of sign compared to the journal version of \cite{Feldmann:2015xsa}, which has a
wrong but inconsequential global factor of $-1$. The operators are defined as
\begin{equation}
\begin{aligned}
    \op{V,L(R)}
        & \equiv \left[\bar{u} \gamma^\mu P_{L(R)} b\right]\,\left[\bar{\ell} \gamma_\mu P_L \nu_\ell\right]\,, \\
    \op{S,L(R)}
        & \equiv \left[\bar{u} P_{L(R)} b\right]\,\left[\bar{\ell} P_L \nu_\ell\right]\,, \\
    \op{T}
        & \equiv \left[\bar{u} \sigma^{\mu\nu} b\right]\,\left[\bar{\ell} \sigma_{\mu\nu} P_L \nu_\ell\right]\,.
\end{aligned}
\end{equation}


%%%%%%%%%%%%
\section{$|\Delta B| = |\Delta C| = 1$}

The Lagrangian reads:
\begin{equation}
\begin{aligned}
  {\cal L}_\text{EFT} &
  = {\cal L}_{\text{QCD} \times \text{QED}} (q;\, \ell) - \frac{4 G_F}{\sqrt{2}} V_{cb}^\text{eff} \left[
      \sum_{X}  \wilson{X} \op{X}
    \right] \,.
\end{aligned}
\end{equation}
Note the change of sign compared to the journal version of \cite{Feldmann:2015xsa}, which has a
wrong but inconsequential global factor of $-1$. The operators are defined as
\begin{equation}
\begin{aligned}
    \op{V,L(R)}
        & \equiv \left[\bar{c} \gamma^\mu P_{L(R)} b\right]\,\left[\bar{\ell} \gamma_\mu P_L \nu_\ell\right]\,, \\
    \op{S,L(R)}
        & \equiv \left[\bar{c} P_{L(R)} b\right]\,\left[\bar{\ell} P_L \nu_\ell\right]\,, \\
    \op{T}
        & \equiv \left[\bar{c} \sigma^{\mu\nu} b\right]\,\left[\bar{\ell} \sigma_{\mu\nu} P_L \nu_\ell\right]\,.
\end{aligned}
\end{equation}
%%%%%%%%%%%%

